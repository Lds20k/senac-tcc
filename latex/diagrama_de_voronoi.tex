\subsection{Diagrama de Voronoi}
\label{subsec:diagrama_de_voronoi}

Segundo \citeonline{rodrigues_diagrama_2019} diagrama de Voronoi é o particionamento do espaço onde cada região é associada a um ponto do conjunto.

O diagrama de Voronoi é gerado a partir das distancias euclidianas entre os vizinhos de um conjunto de pontos do plano\space
\cite{diagrama_de_voronoi:_uma_exploracao_nas_distancias_euclidiana_e_do_taxi}. Esse diagrama possui uma gama de utilizações, por exemplo, estudar epidemias, encontrar o
ponto mais próximo, calcular a precipitação de uma área, estudar os padrões de crescimento das florestas, etc,\space\cite{poligonos_de_thiessen_ou_voronoi}.

Seja um conjunto de índices $I_n = \{1, 2, 3, ..., n\}$ e $A = \{p_1, p_2, ..., p_n\} \subset \mathbb{R}^2$ um conjunto de pontos onde $2 \leq n < \infty$, define-se então como região de Voronoi o conjunto de pontos associado a $p_i$, onde d é a distancia euclidiana

\begin{equation}
	V(p_i) = \{p|d(p_i,p) \leq d(p_j,p);i \neq j, i, j \in I_n\},
\end{equation}

Tem-se um conjunto formado por essas regiões sendo $V(A) = {V(1), V(2), ..., V(n)}$ \space\cite{rodrigues_diagrama_2019}.

Na figura \cref{fig:diagrama_voronoi} pode-se ver a relação dos conjuntos de pontos com o diagrama de Voronoi. Contendo os pontos em vermelhos e retas que são perpendiculares a distância dos pontos vermelhos vizinhos.

\begin{figure}[ht]
	\centering
	\caption{Diagrama de Voronoi.}
	\includegraphics[width=0.6\textwidth]{figures/diagrama_voronoi.png}
	\legend{Fonte: \citeonline{diagrama_de_voronoi_e_suas_aplicacoes_em_sig}}
	\label{fig:diagrama_voronoi}
\end{figure}

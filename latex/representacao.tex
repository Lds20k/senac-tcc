\subsection{Representação}
\label{subsec:representacao}

Existem duas formas de representar os mapas gerados pela geração procedural, uma delas é a representação 2D que possuem duas dimensões de espaço, não possuem perspectiva e são criadas a partir de figuras geométricas — como por exemplo os polígonos gerados pelo diagrama de Voronoi — e imagens \cite{unitywebpage}. Pode-se utilizar os dados da elevação para representar o mapa gerado, onde cada pixel possui um valor de 0 a 255 definidos pela altura, sendo 0 (preto) a menor altura possível e 255 (branco) a maior altura possível, assim tendo um mapa nomeado de mapa de altura. Com o mapa de altura é possível criar uma representação 3D, segundo \citeonline{unitywebpage} os jogos 3D contem três dimensões de espaço, utiliza-se perspectiva e objetos sólidos com geometria tridimensional.

Com a integração de dois mapas distintos, torna-se viável proporcionar uma perspectiva tridimensional em um deles e uma visão superior — conhecida como minimapa — no outro. Esses mapas atuam em conjunto, complementando-se para oferecer ao jogador uma percepção de sua localização. Conforme destacado por \citeonline{microsoft-games-azure}, o Unity se destaca como uma ferramenta versátil para o desenvolvimento de jogos tanto em 2D quanto em 3D, permitindo a representação conjunta de mapas tridimensionais e bidimensionais. Para integrar o Unity com o mapa de elevação existe o pacote \textit{Terrain Tools} que cria um terreno a partir da imagem e o \textit{Terrain Toolkit 2017} que carrega as texturas de acordo com a altura \cite{unity-terrain-tools, unity-terrain-toolkit}.

De acordo com \citeonline{satellasoft}, o Unity possibilita a criação de um personagem que oferece uma interação com o mundo tridimensional do Unity.
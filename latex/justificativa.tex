\section{Justificativa}

O mapa é um elemento que se destaca em jogos digitais e pode ser criado usando técnicas de geração procedural de conteúdo, porém existe um desafio em criar cenários bonitos e diversificados \cite{geracao_procedural_jogos_2d}.

Segundo \citeonline{diagrama_voronoi_jogos}, a área de Geometria Computacional é um ramo da ciência da computação que estuda algoritmos e estruturas de dados para resolução computacional de problemas geométricos e o diagrama de Voronoi é um dos tópicos mais discutidos dessa área. O diagrama de Voronoi pode ser utilizado para resolver alguns problemas relacionados à jogos como por exemplo marcar pontos no mapa e desses pontos criar regiões, a partir dessas regiões criar biomas para serem usados no algoritmo de geração procedural de conteúdo para criar mapas.

De acordo com \citeonline{jogo_procedural} é muito comum em jogos usar técnicas procedurais para otimizar o processo de criação além de ser comum o uso conjunto de inteligência artificial para melhorar ou personalizar como o jogo RimWorld que um simulador conduzido por uma IA que gera histórias no modo procedural.

Dito isso, nosso projeto tem a ideia de fornecer recursos baseados em matemática aplicada dentro de ciência da computação que proporcione uma funcionalidade  de escolher o contorno do mapa no qual irá jogar através de imagens. Abordaremos a arquitetura de redes neurais convolucionais, que é muito utilizada para trabalhar com imagens. Mais especificamente, abordaremos uma arquitetura derivada da anteriormente citada, específica para segmentação de imagens, o que possibilita reconhecer contornos em imagens. Complementando que IA não é a única maneira de encontrar bordas em imagens, contemplaremos outras técnicas específicas de segmentação de imagem.

Adicionando a isso por definição o conceito de ilha é terra cercada de água, logo a única diferença para um continente é o seu tamanho, baseando-se nisso, a escolha de ser uma ilha é porque é uma opção generalista utilizada em diversos jogos como Grand Theft Auto V, Just Cause 4, Fortnite, Pokémon Scarlet \& Violet, dentre outros. Acrescentando sobre a decisão de criar mapas 2D, é pontuado que há uma complexidade enorme entre mapas de 2D e 3D e que não é o foco proposto para o trabalho em questão. 

\section{Inteligência Artificial}

Inteligência artificial é uma técnica científica que simula o pensamento humano de forma que possa ser executado em uma máquina, podendo ser utilizada para criar soluções com uma linha de progressão parecida ao raciocínio lógico como conhecemos. Isto permite ao computador reconhecer e interpretar o mundo ao redor com imagens e textos criando uma ampla área de atuação que otimiza tarefas antes só realizadas por seres humanos \space\cite{ia_aliada_ou_inimiga}.

Este ramo é complexo por se tratar de uma representação cognitiva, se torna necessário usar uma base com diversas áreas científicas como psicologia, biologia, lógica matemática, linguística, engenharia, filosofia, entre outras. E pode ser usado para diversos problemas específicos como, por exemplo, definir as boas rotas para algum processo logístico \space\cite{ia_conceitos_aplicacoes}.

\begin{figure}[H]
	\caption{Diagrama de aprendizado de máquina}
	\centering % para centralizarmos a figura
	\includegraphics[width=10cm]{figures/areas_ia.png} % leia abaixo
	\legend{Fonte: \citeonline{aplicacoes_ia_vg}}
	\label{fig:areas_ia}
\end{figure}

Segundo \citeonline{dp_overview} existe três tópicos sobre inteligência artificial muito populares sendo eles, inteligência artificial, aprendizado de máquina e aprendizado profundo como segue na imagem \cref{fig:diagrama_ia_ml_dp}.

\begin{figure}[H]
	\caption{Diagrama de Venn sobre relação entre os tópicos de inteligência artificial}
	\centering % para centralizarmos a figura
	\includegraphics[width=10cm]{figures/diagrama_ia_ml_dp.png} % leia abaixo
	\legend{Fonte: \citeonline{dp_overview}}
	\label{fig:diagrama_ia_ml_dp}
\end{figure}


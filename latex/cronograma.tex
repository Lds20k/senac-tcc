\section{Cronograma}

O processo de desenvolvimento será separado em 3 tópicos principais, inteligencia artificial, diagrama de Voronoi e interface de usuário. O desenvolvimento de cada tópico do software será feito em paralelo, pois os tópicos não possuem acoplamento.

\subsection*{Inteligencia Artificial}

Primeiro será necessário desenvolver o código da rede neural utilizando \textit{TensorFlow}, especificar a quantidade de camadas e definir o tamanho de entrada e saída das imagens.
O proximo passo será buscar um dataset para fazer o treinamento desse modelo e em seguida validar as saídas.

O tempo estimado para o desenvolvimento é de 1 a 2 meses, a maior parte será para validar o resultado do treinamento.

\subsection*{Diagrama de Voronoi}

Para o desenvolver código do diagrama de Voronoi será preciso primeiro gerar os pontos e desses pontos as áreas, fazer o algoritmo entender se a área tocou no segmento de imagem, caso tenha tocado armazenar para um processamento posterior que irá especificar qual bioma aquela áreas será, para fazer os teste será necessário uma imagem com um polígono.

O tempo estimado para o desenvolvimento é de 1 mes.

\subsection*{Interface de Usuário}

A interface de usuário terá 5 telas principais, inicio, processamento da segmentação, seleção, processamento de seleção, resultado. 

As telas terão o seguinte fluxo:

\begin{figure}[H]
	\centering
    \caption{Tela de inicio, botões de carregar imagem e carregar projeto, menu de contexto arquivos com 3 botões, carregar imagem, carregar projeto e salvar.}
	\includegraphics[width=0.55\textwidth]{figures/tela_novo.png}
    \legend{Fonte: Criação própia}
	\label{fig:tela_novo}
\end{figure}


\begin{figure}[H]
	\centering
    \caption{Tela de processamento da segmentação}
	\includegraphics[width=0.55\textwidth]{figures/tela_processando_1.png}
    \legend{Fonte: Criação própia}
	\label{fig:tela_processando_1}
\end{figure}


\begin{figure}[H]
	\centering
    \caption{Tela de seleção de segmentação da imagem.}
	\includegraphics[width=0.8\textwidth]{figures/tela_carregado.png}
    \legend{Fonte: Criação própia}
	\label{fig:tela_carregado}
\end{figure}


\begin{figure}[H]
	\centering
    \caption{Tela de processamento para geração do mapa com a seleção do segmento.}
	\includegraphics[width=0.8\textwidth]{figures/tela_processando_2.png}
    \legend{Fonte: Criação própia}
	\label{fig:tela_processando_2}
\end{figure}


\begin{figure}[H]
	\centering
    \caption{Tela de resultado com o mapa gerado após processamento.}
	\includegraphics[width=0.6\textwidth]{figures/tela_mapa.png}
    \legend{Fonte: Criação própia}
	\label{fig:tela_mapa}
\end{figure}

Após isso a interface permitira o usuário salvar o projeto bem como exportar o resultado.

O tempo de desenvolvimento será em torno de 1 mês.

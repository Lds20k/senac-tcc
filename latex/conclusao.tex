\section{Conclusão}

A presente monografia aborda uma solução que viabiliza a criação de mapas 2D e 3D a partir de contornos de dois contextos distintos: uma imagem de um desenho, utilizando seleção por preenchimento por inundação, e uma imagem urbana, empregando segmentação panóptica. A implementação do projeto compreende seis fases distintas: a primeira fase concentra-se na segmentação de imagens; a segunda, na seleção de contornos; a terceira fase envolve a geração procedural de mapas com biomas variados; a quarta etapa consiste no desenvolvimento de testes para avaliar a eficácia da geração procedural; a quinta fase aborda a análise de casos de pós-processamento; e, por fim, a sexta etapa concentra-se na integração das ferramentas mencionadas por meio de uma interface gráfica intuitiva.

Os trabalhos relacionados desempenharam um papel crucial para a conclusão desta monografia. A contribuição de Kirillov (2019) foi essencial para o entendimento e a base da segmentação panóptica, que combina a segmentação de instância e semântica, além de fornecer fundamentos para a técnica de classificação de conjuntos, permitindo a utilização de métricas para validar a hipótese inicial. O modelo eficiente de segmentação panóptica descrito por Mohan et al. (2020) foi utilizado para segmentar as imagens e selecioná-las, viabilizando a criação procedural de mapas para jogos, conforme descrito por Amit (2010), com o contorno escolhido.

A fundamentação teórica proporcionada pelos trabalhos relacionados possibilitou a construção do protótipo e a execução dos testes propostos. Além disso, a proposta inicial foi comprovada, indicando que um aumento nos pontos do diagrama de Voronoi para a geração procedural do mapa resulta em uma maior aproximação do contorno da ilha gerado a partir das técnicas de segmentação.

Conclui-se que todos os objetivos propostos foram alcançados de maneira satisfatória, obtendo resultados positivos em ambos os cenários. A diversificação e personalização dos ambientes virtuais promovem um impacto positivo na experiência do jogador. Contudo, é importante ressaltar que a relação entre a adição de mais pontos e a melhoria dos resultados, juntamente com o tempo de execução, pode apresentar desafios em dispositivos com capacidades de processamento inferiores. Além disso, a dependência da biblioteca CUDA Toolkit na implementação da inteligência artificial pode limitar sua aplicabilidade em cenários menos ideais.

Nesse contexto, sugere-se a possibilidade de aprimorar a proposta utilizando algoritmos mais eficientes e multiplataforma, ampliando assim a funcionalidade para um público mais abrangente. A automação para Unity revela-se versátil, podendo ser adotada por desenvolvedores para a criação rápida de esboços de mapas 2D/3D ou como funcionalidade para consumidores de jogos, permitindo a geração de mapas a partir de contornos selecionados em fotos segmentadas.

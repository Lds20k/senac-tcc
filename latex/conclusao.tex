\section{Conclusão}

Esta monografia apresentou uma solução capaz de cria mapas 2d e 3d a partir de um contorno de uma imagem utilizando segmentação panóptica. Pode-se dividir em 4 fases sendo elas: segmentação da imagem, seleção de um contorno, geração procedural dos mapas e a automação para o terreno no Unity.

Portanto pode-se afirmar que o objetivo do trabalho foi concluído com um resultado satisfatório. Porém pelo fato de existir uma proporção entre adicionar mais pontos e melhorar os resultados junto com o tempo de execução, pode ser um problema em algum cenário com um equipamento com processamento inferior. Além do fato do fato da inteligência artificial ter sido desenvolvimento usando a biblioteca CUDA Toolkit que pode limitar ainda mais o devido as condições do cenário ideal.

Com esse contexto, a proposta apresentada pode ser usada para resultados melhores, provavelmente utilizando algoritmos mais eficientes e multiplataforma para permitir essa funcionalidade para mais usuários. Com a automação para Unity mostra-se as duas funcionalidades podendo ser utilizado por desenvolvedores para criarem um esboço rápido de um mapa 2d/3d ou como funcionalidade para o consumidor de jogos, abrindo a possibilidade de gerar um mapa a partir de um contorno selecionado de uma foto tirada e segmentada.
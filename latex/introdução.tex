% introduzindo a jogos, por que é um mercado que está tao em alta
A indústria de jogos digitais cresce cada vez mais. De acordo com \citeonline{quanto_games_vao_movimentar}, essa indústria tende a ultrapassar em 2023, os US\$ 200 bilhões (aproximadamente, R\$ 1 trilhão). Novos jogos são produzidos e publicados diariamente, e somente na plataforma digital Steam, foram 10.644 novos títulos em 2022 como podemos ver na \cref{fig:steam_publishes} \space
% \cite{número_de_jogos_publicados_na_steam}.

\begin{figure}[!ht]
	\centering
    \caption{número de jogos publicados na Steam.}
	\includegraphics[width=0.6\textwidth]{figures/steam_sales.png}
	\legend{Fonte: \citeonline{numero_de_jogos_publicados_na_steam}}
	\label{fig:steam_publishes}
\end{figure}


% aqui a gente aproveita que falou de jogos para introduzir MAPAS que é o 'tema' do tcc
No cenário de jogos, os mapas desempenham um papel fundamental, fornecendo orientação aos jogadores e criando a sensação de escala em uma área. Por exemplo o jogo de aventura pirata chamado Sea of Thieves, os mapas revelam locais de interesse, como tesouros escondidos, missões e áreas perigosas, além de ajudar os jogadores a planejar suas estratégias, explorar o mundo virtual e tomar decisões com base em informações espaciais. Portanto os mapas enriquecem a experiência geral do jogo, mas cria-los pode ser um desafio, especialmente levando em consideração o orçamento disponível. Pois demandaria muitos recursos criar vários mapas diferentes com intuito de entretenimento do jogador. Em jogos como Minecraft, um elemento importante é a geração procedural, que consiste em um conjunto de algoritmos e ferramentas para geração de conteúdo, no qual se cria os mundos, com ilhas contendo biomas, cavernas, vilas, dentre outros recursos. Com essa diversidade de características pode-se evitar o tédio de sempre jogar no mesmo mapa \space\cite{video-game-maps, lecafedugeek}.

% \begin{figure}[ht]
% 	\caption{Mapa de tesouro do jogo Sea of Thieves}
% 	\centering % para centralizarmos a figura
% 	\includegraphics[width=10cm]{figures/Treasure_Map.jpg} % leia abaixo
% 	\legend{Fonte: \citeonline{seaofthieves}}
% 	\label{fig:treasureMap}
% \end{figure}

% aqui a gente faz um adendo e a demanda de jogos tende a crescer, então da a entender que você tem que produzir cada vez mais
% e também fala o quanto custa para produzir um jogo
% Ademais, o mercado de jogos no Brasil teve um aumento de 2,5\% em 2022, como apontado por uma pesquisa sobre o crescimento da demanda \space \space\cite{pesquisa_games_brasil}. O custo de produção de jogos varia bastante, dependendo do tamanho e da complexidade do projeto, \emph{e.g.}, a empresa Rockstar Games revelou que o jogo \textit{Grand Theft Auto V} custou cerca de 265 milhões de dólares para ser desenvolvido e comercializado \space \space\cite{gta_quanto_custou}.

%solução para o problema
% Apesar do rápido crescimento da indústria, existe uma carência de ferramentas que possam auxiliar os designers e artistas durante o processo de produção de jogos, o que acaba tornando-o demorado e, consequentemente, mais caro.  Segundo o livro "Procedural Content Generation in Games" \space\cite{procedural_centent_book}, uma abordagem eficiente para reduzir os custos de produção de um jogo é utilizar a geração procedural de conteúdo. Essa técnica permite maximizar o desenvolvimento de um jogo, envolvendo o uso de um software de computador capaz de criar conteúdo de jogos automaticamente. Esse software possibilita a geração automatizada de mapas, otimizando o processo de desenvolvimento.


% No entanto, a criação de mapas usando esse método ainda encontram dificuldades, sendo elas, variedade e autenticidade \space\cite{geracao_procedural_jogos_2d}.
% aqui adicionar uma explicação do porque é um desafio a geração procedural de conteúdo

Contextualizando, a área de Geometria Computacional é um ramo da ciência da computação que estuda algoritmos e estruturas de dados, servindo para resolução computacional de problemas geométricos. O diagrama de Voronoi é um dos tópicos mais discutidos dessa
área e possui uma gama de utilizações, dentre elas pode ser utilizado para resolver alguns problemas relacionados a jogos como por exemplo marcar pontos no mapa, desses pontos criar regiões, e a partir dessas regiões criar biomas gerando um mapa \space\cite{rodrigues_diagrama_2019}.


% introduz a relação de ia para personalização dentro de métodos procedurais em jogos
De acordo com \citeonline{jogo_procedural} é muito comum usar técnicas procedurais combinado com Inteligência Artificial (IA) para melhorar ou personalizar a experiência do jogador. Por exemplo, o jogo RimWorld é um simulador de colônia que gera um planeta de forma procedural e utiliza uma IA para narrar a história, abrangendo psicologia, ecologia, combate e diplomacia, dentre outros. Logo, essa combinação entre IA e a geração procedural cria uma jogabilidade única ao jogador.

% A aplicação da IA em jogos não se limita apenas à jogabilidade. Ela também é usada em áreas como animação de personagens, reconhecimento de fala e expressões faciais, tradução automática de idiomas nos diálogos do jogo e muito mais. A IA está impulsionando a inovação e a evolução dos jogos, proporcionando experiências cada vez mais envolventes e cativantes para os jogadores \space\cite{exameNvidia, omniverseace}.

% introduz o ramo de segmentação geral que será explicado mais para frente
Em IA, um ramo que está em ascensão é o de segmentação de imagem com redes neurais convolucionais, que constitui-se em classificar os pixeis de uma imagem ou criar áreas na imagem para destacar cada objeto (todas classes que são contáveis como pessoas, carros, etc) detectado ou até mesmo mesclar essas duas técnicas. Neste ramo existem diversas aplicações, como por exemplo carros autônomos e sistemas de vigilância.
Na aplicação de carros autônomos é necessário identificar humanos para tomar decisões de freio, em sistemas de vigilância é necessário identificar para alertar e automatizar o processo de segurança. Portanto, nessas aplicações reais observa-se a importância em identificar seres humanos para a tomada de decisões \space\cite{dp_semantic_segmantation}.

% Nessas aplicações é possível observar que é preciso ter um foco em identificar e segmentar seres humanos, por exemplo, em carros autônomos é primordial essa tarefa para o carro tomar a decisão de frear quando estiver muito perto de bater.

% Logo, se torna um tópico relevante dentro de visão computacional, no qual pode ter diversas aplicações no mundo real \space\cite{kirillov2019panoptic, dp_semantic_segmantation}.

Com base na contextualização é possível perceber que o mercado de jogos está em ascensão, o mapa é um recurso importante e pode ser usado a técnica de geração procedural para diversificar, o diagrama de Voronoi pode ser usado para gerar biomas em mapas no processo de geração procedural, a técnica de geração procedural de conteúdo unido a inteligência artificial é muito utilizado em jogos para criar personalizações, no ramo de inteligência artificial a segmentação com redes neurais convolucionais está em destaque. Logo pode-se perceber a relevância desses temas no curso de ciência da computação e na atualidade, propõe-se então, uma solução para personalizar mapas de jogos utilizando um modelo de IA da área de segmentação usando redes neurais convolucionais.

Com o objetivo de gerar mapas com biomas de forma procedural com personalização de IA, decidiu-se utilizar o resultado da segmentação de imagem por rede neural convolucional para o usuário selecionar uma área e assim delimitar o contorno da ilha, adicionando, portanto, uma personalização. Essa aplicação possibilita um desenvolvedor de jogos criar um protótipo de mapa rapidamente ou aprimorar e usar esse recurso no jogo, possibilitando o jogador tirar ou selecionar uma foto e escolher um contorno para gerar um mapa com aquele formato.

Por fim, para contribuição científica tem-se a hipótese de que quanto mais pontos o diagrama de Voronoi tiver maior será a precisão da compatibilidade entre o mapa gerado e o contorno escolhido. Para chegar a essa conclusão, comprometeu-se definir alguns testes com métricas em prol de mensurar a qualidade da geração procedural com o contorno selecionado.


% Por conseguinte, a combinação entre inteligência artificial e geração procedural de mapas pode abrir novas possibilidades de personalização nos jogos. Imagine um jogo em que, a partir da segmentação de imagens por meio de redes neurais convolucionais, os jogadores possam criar mapas únicos e personalizados para suas aventuras. Com uma foto, o modelo treinado segmentaria a imagem para selecionar um contorno reconhecido, e a partir dele se criar um mapa de maneira procedural contendo biomas no mesmo formato escolhido.

% No contexto da geração procedural de mapas, explorar a relação entre IA e personalização de jogos contribuirá para o avanço dessas áreas de pesquisa, proporcionando aos jogadores experiências mais ricas e variadas.

% Adicionar uma parte explicando a parte de visão computacional e porque o tema da nossa Ia é identificação de pessoas

% Dito isso, nosso projeto tem a ideia de fornecer recursos baseados em matemática aplicada dentro de ciência da computação que proporcione uma funcionalidade de escolher o contorno do mapa no qual irá jogar através de imagens. Abordaremos a arquitetura de redes neurais convolucionais, que é muito utilizada para trabalhar com imagens. Mais especificamente, abordaremos uma arquitetura derivada da arquitetura mencionada anteriormente, específica para segmentação de imagens, o que possibilita classificar contornos em imagens.

\section{Objetivos}

O objetivo principal deste trabalho é desenvolver uma ferramenta que ofereça uma alternativa para a geração procedural de mapas de ilhas, utilizando o diagrama de Voronoi para a criar biomas. Além disso, pretende-se combinar segmentação com redes neurais convolucionais para permitir a personalização desses mapas. Essa ferramenta terá a capacidade de reconhecer os contornos reconhecidos (classificados no conjunto de dados, logo o resultado terá uma detecção abrangente dentro do escopo de classes obtidas) de uma imagem, e gerar um mapa com um mapa baseado nos limites do contorno escolhido.

Adicionalmente, os seguintes objetivos específicos serão abordados:

\begin{itemize}
	\item Selecionar e analisar conjuntos de dados contendo classes relevantes, como pessoas, carros, entre outros, para treinar um modelo de rede neural convolucional específico para segmentação de imagens.
	\item Utilizar algoritmos para criar diagramas de Voronoi.
	\item Aplicar um algoritmo para reconhecer a imagem com o contorno selecionado e gerar como resultado a imagem do mapa gerado.
	\item Utilizar o resultado da segmentação para selecionar indicar o que é terreno em cima do diagrama de Voronoi.
	\item Gerar os biomas no diagrama de Voronoi.
	\item Criar testes em prol de mensurar a semelhança entre o contorno do mapa gerado com o contorno escolhido.
\end{itemize}

% Outro cenário que está crescendo muito nos últimos anos é o da inteligência artificial, afirma \citeonline{Valente_2020} que no Brasil mais que dobrou o número contratações de desenvolvedores da área de 2015 até 2020. De acordo com \apud{johnson2023}{briggs2023} um relatório recente relata que 300 milhões de empregos podem ser afetados pela IA \emph{i.e.} 18\% ofício global pode ser automatizado. Outrossim \citeonline{europarl2020} diz que o tópico de inteligência artificial é uma prioridade para União Europeia por ser considerada primordial para transformação digital da sociedade.  Do mesmo modo, Bill Gates, um dos fundadores da Microsoft — uma das maiores empresas de tecnologia —, diz que "o desenvolvimento da inteligência artificial (IA) é o avanço tecnológico mais importante em décadas"\space
% \space\cite{inteligencia_artificial_e_avanco_bbc}.
% introduzindo a jogos, por que é um mercado que está tao em alta
A indústria de jogos digitais cresce cada vez mais a cada dia, de acordo com a consultoria Newzoo \space\cite{quanto_games_vao_movimentar}, essa indústria tende a ultrapassar em 2023, os US\$ 200 bilhões (aproximadamente, R\$ 1 trilhão). Novos jogos são produzidos e publicados diariamente, e somente na plataforma digital Steam, foram 10.963 novos títulos em 2022\space
\cite{numero_de_jogos_publicados_na_steam}.

% aqui a gente aproveita que falou de jogos para introduzir MAPAS que é o 'tema' do tcc
Os mapas desempenham um papel fundamental nos jogos, fornecendo orientação aos jogadores e criando a sensação de escala em uma área. Por exemplo o jogo de aventura pirata chamado Sea of Thieves, os mapas revelam locais de interesse, como tesouros escondidos, missões e áreas perigosas. Na \cref{fig:treasureMap} pode-se observar um exemplo de mapa dentro do jogo. Eles ajudam os jogadores a planejar suas estratégias, explorar o mundo virtual e tomar decisões com base em informações espaciais. Além disso, os mapas podem transmitir a sensação de escala e proporção, dando aos jogadores uma compreensão visual da extensão do mundo do jogo. Portanto os mapas enriquecem a experiência geral do jogo, mas cria-los pode ser um desafio, especialmente levando em consideração o orçamento disponível, uma vez que um jogo para chegar no consumidor final, passa por diversos processos de criação, o que demanda muito tempo e dinheiro \cite{video-game-maps, lecafedugeek}.

\begin{figure}[H]
	\caption{Mapa de tesouro do jogo Sea of Thieves}
	\centering % para centralizarmos a figura
	\includegraphics[width=10cm]{figures/Treasure_Map.jpg} % leia abaixo
	\legend{Fonte: \citeonline{seaofthieves}}
	\label{fig:treasureMap}
\end{figure}

% aqui a gente faz um adendo e a demanda de jogos tende a crescer, então da a entender que voce tem que produzir cada vez mais
% e tambem fala o quanto custa para produzir um jogo
Ademais, o mercado de jogos no Brasil teve um aumento de 2,5\% em 2022, como apontado por uma pesquisa sobre o crescimento da demanda \space \cite{pesquisa_games_brasil}. O custo de produção de jogos varia bastante, dependendo do tamanho e da complexidade do projeto, \emph{e.g.}, a empresa Rockstar Games revelou que o jogo \textit{Grand Theft Auto V} custou cerca de 265 milhões de dólares para ser desenvolvido e comercializado \space
\cite{gta_quanto_custou}.

%solucao para o problema 
Uma abordagem eficiente para reduzir os custos de produção de um jogo é utilizar a geração procedural de conteúdo. Essa técnica envolve o uso de um software de computador capaz de criar conteúdo de jogos por conta própria \cite{procedural_centent_book}. Esse software permite a geração automatizada de mapas, tornando o processo de desenvolvimento otimizado.

% No entanto, a criação de mapas usando esse método ainda encontram dificuldades, sendo elas, variedade e autenticidade \cite{geracao_procedural_jogos_2d}.
% aqui adicionar uma explicação do porque é um desafio a geração procedeural de conteudo 

% introduz a relação de ia para personalização dentro de métodos procedurais em jogos
De acordo com \citeonline{jogo_procedural} é muito comum usar técnicas procedurais em jogos para otimizar o processo de criação combinado com inteligência artificial para melhorar ou personalizar a experiência do jogador. Por exemplo, o jogo RimWorld é um simulador de colônia que utiliza uma IA para gerar histórias de forma procedural, abrangendo aspectos como psicologia, ecologia, combate e diplomacia, dentre outros \cite{jogo_procedural}.

A aplicação da IA em jogos não se limita apenas à jogabilidade. Ela também é usada em áreas como animação de personagens, reconhecimento de fala e expressões faciais, tradução automática de idiomas nos diálogos do jogo e muito mais. A IA está impulsionando a inovação e a evolução dos jogos, proporcionando experiências cada vez mais envolventes e cativantes para os jogadores \cite{exameNvidia, omniverseace}.

% introduz o ramo de segmentação geral que será explicado mais para frente
Outro ramo de IA que está em ascensão é o de segmentação de imagem com redes neurais convolucionais, onde é possível classificar os pixeis de uma imagem além de ser possível criar uma máscara  para cada objeto\footnote{Todas classes que são contáveis como pessoas, carros, etc.} detectado. As suas aplicações são diversas como por exemplo, carros ou drones autônomos, sistemas de vigilância, sistemas militares inteligentes, entre outros. Nessas aplicações é possível observar que é preciso ter um foco em identificar e segmentar seres humanos, por exemplo em carros autônomos é primordial essa tarefa para o carro tomar a decisão de frear quando estiver muito perto. Logo, se torna um tópico relevante dentro de visão computacional que pode ter diversas aplicações no mundo real, tal como segmentar uma imagem para detectar contornos e a partir disso criar um mapa personalizado \cite{kirillov2019panoptic, dp_semantic_segmantation}.

No contexto da geração procedural de mapas, explorar a relação entre IA e personalização de jogos contribuirá para o avanço dessas áreas de pesquisa, proporcionando aos jogadores experiências mais ricas e variadas.

% Adicionar uma parte explicando a parte de visão computacional e porque o tema da nossa Ia é identificação de pessoas 

% Dito isso, nosso projeto tem a ideia de fornecer recursos baseados em matemática aplicada dentro de ciência da computação que proporcione uma funcionalidade de escolher o contorno do mapa no qual irá jogar através de imagens. Abordaremos a arquitetura de redes neurais convolucionais, que é muito utilizada para trabalhar com imagens. Mais especificamente, abordaremos uma arquitetura derivada da arquitetura mencionada anteriormente, específica para segmentação de imagens, o que possibilita classificar contornos em imagens.

\section{Objetivos}

O objetivo principal deste trabalho é desenvolver uma ferramenta que ofereça uma alternativa para a geração procedural de mapas 2D de ilhas, utilizando o diagrama de Voronoi para a construção dos biomas. Além disso, pretende-se combinar técnicas de segmentação com redes neurais convolucionais para permitir a personalização desses mapas. Essa ferramenta terá a capacidade de reconhecer os contornos\footnote{Os contornos reconhecidos são os classificados no conjunto de dados, logo o resultado terá uma detecção abrangente dentro do escopo de classes obtidas} de uma imagem selecionada e gerar um mapa que preserva fielmente o contorno escolhido.

Adicionalmente, os seguintes objetivos específicos serão abordados:

\begin{itemize}
	\item Selecionar e analisar conjuntos de dados contendo classes relevantes, como pessoas, carros, entre outros, para treinar um modelo de rede neural convolucional específico para segmentação de imagens.
	\item Avaliar o desempenho geral do modelo usando a métrica de avaliação específica para o nicho de segmentação selecionado.
	\item Testar algoritmos de gerar ruídos para criar o mapa.
	\item Aplicar um algoritmo para reconhecer a imagem com o contorno selecionado e gerar como resultado a imagem do mapa gerado.
\end{itemize}

% Outro cenário que está crescendo muito nos últimos anos é o da inteligência artificial, afirma \citeonline{Valente_2020} que no Brasil mais que dobrou o número contratações de desenvolvedores da área de 2015 até 2020. De acordo com \apud{johnson2023}{briggs2023} um relatório recente relata que 300 milhões de empregos podem ser afetados pela IA \emph{i.e.} 18\% ofício global pode ser automatizado. Outrossim \citeonline{europarl2020} diz que o tópico de inteligência artificial é uma prioridade para União Europeia por ser considerada primordial para transformação digital da sociedade.  Do mesmo modo, Bill Gates, um dos fundadores da Microsoft — uma das maiores empresas de tecnologia —, diz que "o desenvolvimento da inteligência artificial (IA) é o avanço tecnológico mais importante em décadas"\space
% \cite{inteligencia_artificial_e_avanco_bbc}.
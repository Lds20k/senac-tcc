\subsubsection*{Retropropagação}

De acordo com \citeonline{brilliant2023backpropagation} o algoritmo geral de retropropagação é:

\begin{enumerate}
    \item Propagação: calcular os pares de entrada-saída $(\overrightarrow{x_d}, y_d)$ — $\overrightarrow{x_d}$ é o vetor de entrada e $y_d$ a saída verdadeira — e guardar os resultados $\hat{y_d}$ — a saída encontrada no treinamento —, $a_j^k$, $o_j^k$ para cada neurônio $j$ na camada $k$, indo da camada de entrada para camada de saída.
    
    \item Retropropagação: Calcular os pares de entrada-saída $(\overrightarrow{x_d}, y_d)$, chegando na fórmula $\frac{\partial{E_d}}{\partial{w_{ij}^k}}$ que é a derivada parcial do erro total. Na representação $E_d$ é a função de perda e $w_{ij}^k$ é o peso  conectado em um neurônio de $k - 1$. Outra forma de representar é $\delta_j^k o_i^{k - 1}$ e suas variáveis são: $\delta_j^k$ representa o erro do neurônio e $o_i^{k - 1}$ representa a saída do neurônio na camada $k -1$. Essa técnica começa na camada de saída e propaga até a última camada escondida.
    
    $$ \frac{\partial{E_d}}{\partial{w_{ij}^k}} = \delta_j^k o_i^{k - 1} $$
    
    \item Combinar gradientes individuais: Uma média simples é feita com todos resultados de $\frac{\partial{E_d}}{\partial{w_{ij}^k}}$ formando assim o gradiente total representado como $\frac{\partial{E(X, \theta)}}{\partial{w_{ij}^k}}$.
    
    $$ \frac{\partial{E(X,\theta)}}{\partial{w_{ij}^k}} = \frac{1}{N} \sum_{d=1}^{N} \frac{\partial{E_d}}{\partial{w_{ij}^k}} $$
    \item Atualiza os pesos: usando $\alpha$ como taxa de aprendizado e o gradiente total $\frac{\partial{E(X, \theta)}}{\partial{w_{ij}^k}}$ temos a seguinte equação.
    
    $$ \Delta w_{ij}^k = -\alpha \frac{\partial{E(X,\theta)}}{\partial{w_{ij}^k}} $$
\end{enumerate}

Observação: basta trocar $w_{ij}^k$ para $b_{ij}^k$ — trocar o peso pelo viés — em todo algoritmo para ajustar o viés do modelo com a técnica de retropropagação.

\section{Visão computacional}
\label{sec:visao_comp}

A visão computacional está em constante avanço, aproximando cada vez mais os computadores da capacidade visual humana. De acordo com Horst Haußecker e Bernd Jähne, no livro "Computer Vision and Applications" \space\cite{comp_vision_and_applications}, a visão computacional é uma área da computação que se dedica à interpretação de imagens por meio de algoritmos e técnicas de processamento de imagens. Essa área abrange a aquisição, processamento e análise de imagens, com o objetivo de extrair informações úteis para resolver problemas específicos.

Segundo Richard Szeliski no livro "Computer Vision: Algorithms and Applications" \space\cite{computer_vision_richard}, nas últimas décadas ocorreram avanços significativos na busca de aproximar a visão computacional da visão humana, porém não obteve total êxito. Isso ocorre porque, enquanto o olho humano enxerga com aparente facilidade as estruturas tridimensionais e suas nuances, a visão computacional depende de técnicas matemáticas altamente precisas para recuperar a forma tridimensional e a aparência dos objetos.

Nas figuras \cref{fig:imagem_a} e \cref{fig:imagem_b}, evidencia-se a notável capacidade de um computador em distinguir, classificar e até mesmo compreender os elementos presentes em uma fotografia.

\begin{figure}[ht]
    \centering
    \begin{minipage}[h]{0.49\textwidth}
      \caption{Algoritmos de detecção facial e de roupas/cabelos por cor localizam e reconhecem pessoas nesta imagem}
      \centering
      \includegraphics[width=0.6\textwidth]{figures/detectacao_de_faces_exemplo.JPG}
	    \legend{Fonte: \citeonline{computer_vision_richard}}
      \label{fig:imagem_a}
    \end{minipage}
    \hfill
    \begin{minipage}[h]{0.49\textwidth}
      \caption{Segmentação de instâncias de objetos pode-se delinear cada pessoa e objeto em uma cena complexa}
      \centering
      \includegraphics[width=0.6\textwidth]{figures/semantic_intance.JPG}
	    \legend{Fonte: \citeonline{instance_segmentation}}
      \label{fig:imagem_b}
    \end{minipage}
  \end{figure}


No entanto, apesar do sucesso no uso dessas técnicas, o computador ainda não consegue oferecer a mesma quantidade de detalhes na explicação de uma imagem como o olho humano. Isso se deve à maior facilidade do computador em compreender linguagem em comparação à visualização. A tarefa de ensinar um computador a ver e descrever com precisão e riqueza de detalhes o que está sendo observado é extremamente complexa \space\cite{computer_vision_richard}.

A visão é um elemento crucial para capacitar a inteligência artificial a realizar diversas tarefas. A fim de replicar a visão humana, é necessário que as máquinas sejam capazes de adquirir, processar, analisar e compreender imagens. \space\cite{como_funciona_visao_computacional}

% Na \cref{fig:comp_vision} podemos ver uma analogia entre a forma como uma imagem é processada pelo cérebro humano e a forma como é processada por um sistema computacional.

% \begin{figure}[!ht]
% 	\centering
% 	\includegraphics[width=0.6\textwidth]{figures/content_Human_Vision.png}
% 	\caption{Visão humana e sistemas de visão computacional processam dados visuais de maneira semelhante \space\cite{content_Human_Vision}.}
% 	\label{fig:comp_vision}
% \end{figure}

No processamento de computação visual, as imagens são adquiridas e representadas como uma matriz 2D de píxeis. Cada pixel corresponde a um ponto na imagem e é representado por um valor numérico que varia de 0 a 255. Esses valores de pixel descrevem a intensidade da cor em uma escala de cinza caso a imagem de entrada esteja em preto e branco, pois se a imagem ter cores do espectro RGB o computador identificará três matrizes de canais referente as cores correspondentes. Dessa forma, um computador interpreta uma imagem como uma ou mais matrizes de números, permitindo que seja analisado e compreendido os detalhes visuais presentes na imagem Um exemplo dessa matriz exemplificado na \cref{fig:comp_vision} do presidente dos Estados Unidos, Abraham Lincoln\cite{mit_video}.

\begin{figure}[ht]
	\caption{Diagrama de dados de píxeis. À esquerda, uma imagem de Lincoln; no centro, os píxeis rotulados com números de 0 a 255, representando sua luminosidade; e à direita, apenas esses números.}
	\centering
	\includegraphics[width=0.6\textwidth]{figures/lincoln_pixel_values.png}
	\legend{Fonte: \citeonline{content_Human_Vision}}
  \label{fig:comp_vision}
\end{figure}

Os algoritmos de visão computacional utilizados atualmente são fundamentados em reconhecimento de padrões. O procedimento consiste em treinar computadores por meio de uma vasta quantidade de dados visuais. Os computadores processam imagens, rotulam os objetos nelas contidos e identificam padrões entre esses objetos \space\cite{content_Human_Vision}.

Esse processo de treinamento e reconhecimento de padrões permite que os computadores identifiquem objetos e compreendam seu contexto visual. Com essa capacidade, o computador consegue realizar tarefas como, por exemplo, reconhecimento facial \cref{fig:imagem_a}.

Em visão computacional é comum usar algumas técnicas para separar objetos de interesse pois a partir disso é possível aplicar alterações em objetos específicos, e para isso utiliza-se um técnica chamada de máscara binária \cite{NVIDIA,Embarcados}.


% \subsection{Máscara binária}
% \label{subsec:mascara_binaria}

A máscara binária — ou imagem binária — contém apenas duas cores, geralmente preto e branco, ou valores 0 e 1. Sendo o branco (ou 1) o objeto em destaque o o preto (ou 0) o fundo \cite{Embarcados}.

Será abordado dois algoritmos com propostas parecidas para selecionar um objeto e destaca-lo em uma imagem binária, sendo eles selecionar imagem por cor e por inundação, ambos utilizando o biblioteca em python OpenCV, uma biblioteca multiplataforma para visão computacional, tendo métodos para auxiliar na manipulação por exemplo de imagens \cite{OpenCV}.

% \subsubsection{Selecionar por cor}
% \label{subsubsec:sel_cor}

O método de selecionar por cor se baseia em pegar a cor específica do clique na imagem e percorrer a imagem comparando a cor alvo com a cor da imagem, caso seja a mesma pinte o mesmo pixel da nova imagem como branco e caso não seja pinte como preto. Uma maneira performática de selecionar por cor é definir um espectro de cores e aplicar uma mascara usando o método \textit{inRange} da biblioteca OpenCV, pode-se escolher apenas uma cor \cite{OpenCVInRange}.

% \subsubsection{Selecionar por preenchimento de inundação}
% \label{subsubsec:sel_floodFill}

O método de  selecionar por preenchimento por inundação é um algoritmo de expansão a partir de um pixel, validando se contém a mesma cor. A implementação inicia uma matriz de zeros com tamanho 2 pixeis maior do que a imagem original. O clique na  imagem será a semente — ou em inglês  seed — e a partir disso o algoritmo começa uma expansão para os pixeis vizinhos — de cima, baixo, esquerda e direita — caso contenha o mesmo valor de cor pinta de cor branca, e refaz com os pixeis marcados  anteriormente até não ter mais o pontos brancos para marcar \cite{OpenCVFloodFill}.

Ademais, existem diversas opções de ferramentas que auxiliam na criação de interfaces gráficas para visualização e manipulação de imagens, fornecendo funções e modelos para criar telas, sendo uma das mais utilizadas o PyQt5 \cite{uniteai2023}.

% \subsection{PyQt5}

O PyQt5 incorpora as bibliotecas Qt em C++, proporcionando recursos para o desenvolvimento multiplataforma. Isso simplifica a criação de interfaces em Python em diversos ambientes, como Windows, Mac, Linux e dispositivos móveis \cite{pyqt5}.









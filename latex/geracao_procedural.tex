\section{Geração procedural de conteúdo}

Segundo \citeonline{yannakakis2018artificial}, a geração procedural de conteúdo constitui métodos e automações utilizados para gerar conteúdo. Essa prática é uma parte importante da inteligência artificial de um jogo e tem sido empregada desde a década de 1980. Essa técnica pode ser aplicada para gerar níveis, mapas, texturas, regras de jogo, histórias, entre outras coisas.

É difícil determinar qual algoritmo foi utilizado para a geração de conteúdo em jogos modernos, pois os códigos-fonte não são facilmente acessíveis. Nos jogos mais antigos, no entanto, os códigos-fonte e as estratégias utilizadas são acessíveis e bem documentados na internet. Geralmente, são empregados algoritmos de geração aleatória, classificáveis como força bruta, e são utilizados para criar estruturas ou mapas, dependendo do tipo de jogo \cite{dormans2010adventures}.
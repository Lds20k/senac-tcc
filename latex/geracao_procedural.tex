\section{Geração procedural de conteúdo}

Segundo \citeonline{yannakakis2018artificial}, em poucas palavras, a geração procedural de conteúdo são métodos e automações utilizados para gerar conteúdos em jogos. A geração procedural de conteúdo também é uma parte importante da inteligência artificial de um jogo e já vem sendo utilizada desde 1980.
Essa técnica pode ser utilizada para gerar níveis, mapas, textura, regras de jogo, historia, entre outras coisas.

É difícil dizer qual algoritmo foi utilizado para geração de conteúdo dos jogos modernos e os códigos fontes não são facilmente acessíveis. Já nos jogos antigos os códigos fontes e as estratégias utilizadas são acessíveis e muito bem documentadas na internet. São geralmente utilizados algoritmos de geração aleatória que podem ser classificados como sendo de força bruta, e são usados para criar estruturas ou mapas dependendo do tipo de jogo \cite{dormans2010adventures}.

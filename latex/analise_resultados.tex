\section{Analise dos resultados}

Percebe-se com os dados das \crefrange{tab:final_input_output_2d}{tab:final_output_2d_output_3d} que os resultados são bem parecidos e que quanto mais pontos forem usados melhor a precisão em todas as méetricas, porém demora mais segundos em média para executar. Portanto deve-se analisar o cenário para o processamento e qual o melhor custo benefício.

Isso se comprova nas ilustrações da última execução — com 300 pontos — de cada combinação de imagens nas \cref{fig:combs_result}, percebe-se que os erros ilustrados com cores vermelha e verde são poucos e os acertos em branco e cinza prevalecem na grande maioria da imagem.
\section{Análise dos Resultados}

Com base nos dados e observações da \cref{tab:final_input_output_2d} e da \cref{tab:final_input_output_3d}, é possível afirmar que a hipótese inicial se comprovou. Isso ocorre porque, quanto mais pontos o diagrama de Voronoi possuir, maior será a compatibilidade com o contorno. Esse fenômeno se dá devido ao aumento no número de polígonos no diagrama; com mais polígonos em um mesmo espaço, a tendência é que o tamanho desses polígonos diminua. Consequentemente, no cálculo para definir os tipos de terreno, há uma tendência à maior precisão em relação ao contorno inicial. Essa afirmação é corroborada pela tendência observada nas métricas das duas tabelas mencionadas, onde as métricas IoU, Acc, F1 e MCC aumentam a cada linha, e as métricas FDR e FNR apresentam uma diminuição no erro encontrado. Outra métrica relevante é a duração da geração procedural, que aumenta proporcionalmente ao número de pontos, indicando uma relação direta com o tempo de processamento. Portanto, é necessário analisar o cenário para o processamento e determinar qual é o melhor custo-benefício.

O resultado satisfatório é evidenciado nas ilustrações da última execução — com 300 pontos — de cada combinação de imagens nas \cref{fig:result_final}. Percebe-se que os erros, ilustrados em cores vermelha e verde, são mínimos, enquanto os acertos, em branco e cinza, prevalecem na grande maioria da imagem.

Além disso, a \cref{tab:final_output_2d_output_3d} mostra, com as observações realizadas, que se mantém a confiabilidade entre a imagem 2D (minimapa) e o mapa de altura (que formará o mapa 3D), garantindo a funcionalidade de localização em tempo real dentro da execução do jogo.

% destacar com logica que a duração maior é para criar o grafo com os biomas e que gerar os mapas é a parte que demora pouco

%  explicar comparando em relação a cada métricas
% Tipo, esse teste foi melhor a acurácia pq...
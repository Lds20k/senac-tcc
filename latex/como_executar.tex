\section{Instalação do Unity Hub}
\label{sec:instalacao_unityhub}

O Unity é um motor gráfico e uma a plataforma que é líder mundial para criar e operar conteúdo 3D interativo em tempo real \cite{unity_get_started}. O Unity Hub é o gerenciador de projetos que utiliza o motor gráfico Unity \cite{unity_download}.

A instalação do Unity Hub foi realizada por meio do serviço de pacotes do Debian. Foi necessário utilizar o superusuário e adicionar o certificado do repositório do Unity, seguindo os comandos apresentados no \cref{lst:unity_certificado} \cite{install_the_unity_hub}. O comando \emph{su} permite que os comandos subsequentes sejam executados com um usuário diferente da sessão; caso não seja informado, é utilizado o usuário \emph{root} ou superusuário. Já o comando \emph{wget} é utilizado para baixar arquivos da internet. Enquanto o comando \emph{gpg} é uma ferramenta que fornece funções de criptografia e assinatura digital usando o padrão OpenPGP. Por fim, o comando \emph{tee} é utilizado para ler o arquivo da entrada padrão e gravar na saída padrão ou em um arquivo \cite{debian_man_pages}.

\begin{lstlisting}[caption={Trecho de código com comando UNIX para adicionar o certificado do repositório do Unity \cite{install_the_unity_hub}},label={lst:unity_certificado},language=Bash,showstringspaces=false]
    su
    wget -qO - https://hub.unity3d.com/linux/keys/public | \
        gpg --dearmor | \
        tee /usr/share/keyrings/Unity_Technologies_ApS.gpg > /dev/null
\end{lstlisting}

Com o certificado do Unity devidamente incluído, o repositório do Unity foi adicionado ao controlador de pacotes do Debian, as informações dos pacotes foram atualizadas e o Unity foi instalado. Os comandos necessários estão detalhados no \cref{lst:unity_repositorio}. O comando \emph{sh} no Debian serve como um link para o Dash, que, por sua vez, atua como um interpretador de comandos para o sistema. Vale ressaltar que o comando \emph{echo} dentro do comando \emph{sh} tem a função de imprimir uma linha de texto no console. Ao incorporar o símbolo \emph{sinal de maior que}, o texto será direcionado para a impressão em um arquivo \cite{debian_man_pages}.

\begin{lstlisting}[caption={Trecho de código com comando UNIX para adicionar o repositório do Unity \cite{install_the_unity_hub}},label={lst:unity_repositorio},language=Bash,showstringspaces=false]
    sh -c 'echo "deb [signed-by=/usr/share/keyrings/Unity_Technologies_ApS.gpg] \
        https://hub.unity3d.com/linux/repos/deb stable main" > \
        /etc/apt/sources.list.d/unityhub.list'
    apt update
    apt-get install unityhub
\end{lstlisting}

\section{Instalação do Miniconda}
\label{sec:instalacao_miniconda}

Conda, uma ferramenta versátil de gerenciamento de pacotes e ambientes, é compatível com sistemas operacionais Windows, macOS e Linux \cite{conda_documentation}.

Miniconda, um instalador leve e gratuito para o Conda, representa uma versão compacta do Anaconda, incluindo apenas o Conda, Python, os pacotes essenciais para ambos, e um número limitado de outros pacotes \cite{miniconda_documentation}.

Para a administração das dependências do projeto, optou-se pelo uso do Miniconda. O procedimento de instalação desse software pode ser visualizado no \cref{lst:miniconda_download} \cite{miniconda_documentation}. O comando \emph{mkdir} é empregado para criar diretórios ou pastas, caso estes não existam. Paralelamente, o comando \emph{wget} é utilizado para baixar arquivos; nesse contexto, ocorre o download do instalador do Miniconda. O comando \emph{bash}, um interpretador GNU Bourne-Again SHell, é empregado para a execução do arquivo do Miniconda. Por fim, o comando \emph{rm} é aplicado para remover arquivos e diretórios, sendo utilizado, neste caso, para excluir o script de instalação do Miniconda \cite{debian_man_pages}.

\begin{lstlisting}[caption={Trecho de código com comando UNIX para baixar o Miniconda \cite{miniconda_documentation}},label={lst:miniconda_download},language=Bash,showstringspaces=false]
    mkdir -p ~/miniconda3
    wget https://repo.anaconda.com/miniconda/Miniconda3-latest-Linux-x86_64.sh \
        -O ~/miniconda3/miniconda.sh
    bash ~/miniconda3/miniconda.sh -b -u -p ~/miniconda3
    rm -rf ~/miniconda3/miniconda.sh
\end{lstlisting}

Após a instalação do Miniconda, torna-se necessário ativar o ambiente do conda para possibilitar a interação com o shell, conforme descrito na documentação do Conda \cite{conda_documentation}. A abordagem para realizar essa ativação é ilustrada no trecho de código apresentado no \cref{lst:inicializar_conda}. O comando \emph{conda init} é empregado com o propósito de inicializar o conda nos shells bash (interpretador GNU Bourne-Again SHell) e zsh (interpretador de comandos UNIX) \cite{debian_man_pages,conda_documentation}.

\begin{lstlisting}[caption={Trecho de código com comando para inicializar o Miniconda \cite{miniconda_documentation}},label={lst:inicializar_conda},language=Bash,showstringspaces=false]
    ~/miniconda3/bin/conda init bash
    ~/miniconda3/bin/conda init zsh
\end{lstlisting}

\section{Repositório do projeto}
\label{sec:repositorio_projeto}

Todas as informações pertinentes aos códigos, texto e anotações deste trabalho encontram-se disponíveis no repositório online hospedado no GitHub, acessível por meio do link \url{https://github.com/Lds20k/senac-tcc}.

\section{Execução do projeto}

Para a realização do projeto, foi organizado um repositório contendo todas as dependências. Para obter o projeto mencionado na \cref{sec:repositorio_projeto}, recomenda-se utilizar o sistema de controle de versões distribuído Git \cite{git_page}. Através do comando indicado na \cref{lst:git_clone_projeto}, é possível efetuar essa operação. O comando \emph{git clone} realiza o download do projeto a partir de um repositório remoto \cite{git_clone}.

\begin{lstlisting}[caption={Trecho de código com comando do git para baixar o projeto},label={lst:git_clone_projeto},language=Bash,showstringspaces=false]
    git clone https://github.com/Lds20k/senac-tcc.git
\end{lstlisting}

Após ter realizado o download do projeto e instalado as dependências mencionadas nos apêndices \ref{sec:instalacao_unityhub} e \ref{sec:instalacao_miniconda}, é possível configurar o ambiente para a execução do projeto. Para realizar essa configuração, acesse o diretório do projeto e siga os comandos indicados no \cref{lst:conda_dependencias}.

O comando \emph{conda env create} é utilizado para criar um novo ambiente, incorporando as dependências especificadas no arquivo \emph{environment.yml}. Posteriormente, o comando \emph{conda activate} é empregado para ativar o ambiente denominado \emph{senac-tcc}. Adicionalmente, o comando \emph{conda install} é empregado para instalar outras dependências que não estão incluídas no arquivo \emph{environment.yml}. Por fim, o comando \emph{pip install} é utilizado para instalar módulos específicos para o Python, conforme listado no arquivo \emph{requirements} \cite{conda_env_create,conda_deep_dives_activation,conda_install,pip_install}. Este conjunto de ações possibilita a adequada configuração do ambiente, preparando-o para a execução bem-sucedida do projeto.

\begin{lstlisting}[caption={Trecho com instruções para instalação de dependências utilizando conda e pip \cite{efficientpsGit}},label={lst:conda_dependencias},language=Bash,showstringspaces=false]
    conda env create -n senac-tcc --file=environment.yml
    conda activate senac-tcc
    conda install pytorch==1.7.0 torchvision==0.8.0 torchaudio==0.7.0 \
        cudatoolkit=10.2 cudatoolkit-dev -c pytorch -c conda-forge
    pip install -r requirements.txt
\end{lstlisting}

Após a configuração adequada das dependências e do ambiente, o próximo passo consiste na instalação da implementação do EfficientNet e do EfficientPS. Para realizar essa tarefa, é necessário seguir os comandos apresentados no \cref{lst:instalar_efficientpsnet}.

O comando \emph{cd} é utilizado para alterar o diretório de trabalho para o caminho especificado como primeiro argumento. Por sua vez, o comando \emph{python} é empregado para executar um script Python que realiza tanto a compilação quanto a instalação do EfficientNet e do EfficientPS, respectivamente \cite{cd_3tcl,efficientpsGit}.

Ao seguir essas instruções, a implementação desses programas será instalada de maneira apropriada no ambiente configurado, proporcionando os recursos necessários para o desenvolvimento e execução bem-sucedidos do projeto.

\begin{lstlisting}[caption={Trecho de código com comandos para instalação da EfficientNet e EfficientPS \cite{efficientpsGit}},label={lst:instalar_efficientpsnet},language=Bash,showstringspaces=false]
    cd src/efficientNet
    python setup.py develop
    cd ..
    python setup.py develop
\end{lstlisting}

Realize o download do modelo pré-treinado \emph{KITTI} disponível no repositório através do seguinte endereço:

\url{https://github.com/DeepSceneSeg/EfficientPS#pre-trained-models}

Após o download, renomeie o arquivo baixado para \emph{model\_kt.pth} e, em seguida, mova o arquivo do modelo pré-treinado para a pasta \emph{src}. Certifique-se de que o arquivo esteja localizado corretamente dentro do diretório mencionado antes de prosseguir com o uso do modelo no projeto. Essa ação garante a correta integração do modelo pré-treinado no ambiente de desenvolvimento.

Finalmente, para a execução do projeto, é crucial observar a estrutura da pasta de execução, uma vez que o diretório de execução deve iniciar obrigatoriamente a partir da raiz do projeto. No trecho apresentado no \cref{lst:execucao_projeto}, encontra-se o comando para a execução do projeto.

Certifique-se de que o diretório de execução está corretamente configurado para iniciar a partir da raiz do projeto antes de executar o comando mencionado. Isso garantirá que o projeto seja iniciado corretamente, utilizando os recursos e dependências configurados durante o processo anterior.

\begin{lstlisting}[caption={Trecho de código com comando para execução do projeto},label={lst:execucao_projeto},language=Bash,showstringspaces=false]
    python src/main.py
\end{lstlisting}

Para utilizar o projeto no Unity Hub, proceda da seguinte maneira: abra o projeto \emph{3d\_map} localizado dentro da pasta \emph{unity}, na raiz do projeto. Durante o processo, assegure-se de realizar a instalação da versão recomendada das dependências, conforme indicado pela plataforma Unity Hub. Este procedimento garantirá a correta configuração do ambiente de desenvolvimento, possibilitando a utilização do projeto no Unity Hub.

Depois de abrir o projeto no Unity Hub, acesse e abra a cena localizada dentro da pasta \emph{Scenes}. Agora, basta pressionar o botão \emph{play}, e o script se encarregará de carregar as imagens e arquivos gerados pelo programa de geração de mapas. 

No caso de ocorrer um erro relacionado ao módulo Python do Numpy, você pode corrigi-lo desinstalando e reinstalando o módulo. O procedimento está detalhado no \cref{lst:reinstalacao_numpy}. O comando \emph{pip uninstall} remove um pacote previamente instalado, enquanto o \emph{pip install} realiza a instalação de um pacote \cite{pip_install, pip_uninstall}.

\begin{lstlisting}[caption={Trecho de código com os comandos para reinstalação do Numpy},label={lst:reinstalacao_numpy},language=Bash,showstringspaces=false]
    pip uninstall numpy
    pip install numpy
\end{lstlisting}

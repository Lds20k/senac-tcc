\section{Contexto}

A indústria de jogos digitais cresce mais a cada dia, segundo a consultora Newzoo \space\cite{quanto_games_vao_movimentar} essa indústria tende a ultrapassar em 2023 os US\$ 200 bilhões (aproximadamente R\$ 1 trilhão). Novos jogos são produzidos e publicados diariamente e somente na plataforma digital Steam, foram publicados 10.963 novos títulos em 2022\space
\cite{numero_de_jogos_publicados_na_steam}.

Ademais, as empresas de desenvolvimento de jogos continuam a trabalhar incessantemente para atender a uma demanda de mercado que cresceu 2,5\% no Brasil em 2022 
\space
\cite{pesquisa_games_brasil}. O custo de produção de jogos varia bastante, dependendo do tamanho e da complexidade do projeto, \emph{e.g.}, a empresa Rockstar Games revelou que o jogo \"Grand Theft Auto V\" custou cerca de 265 milhões de dólares para ser produzido e comercializado \space
\cite{gta_quanto_custou}.

Outro cenário que está crescendo muito nos últimos anos é o da inteligência artificial afirma \citeonline{Valente_2020} que no Brasil mais que dobrou o número contratações de desenvolvedores da área de 2015 até 2020. De acordo com \apud{johnson2023}{briggs2023} um relatório recente relata que 300 milhões de empregos podem ser afetados pela IA \emph{i.e.} 18\% ofício global pode ser automatizado. Outrossim \citeonline{europarl2020} diz que o tópico de inteligência artificial é uma prioridade para União Europeia por ser considerada primordial para transformação digital da sociedade.  Do mesmo modo, "Bill Gates, um dos fundadores da Microsoft — uma das maiores empresas de tecnologia —, diz que o desenvolvimento da inteligência artificial (IA) é o avanço tecnológico mais importante em décadas"\space
\cite{inteligencia_artificial_e_avanco_bbc}.

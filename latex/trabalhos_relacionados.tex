\section{Trabalhos relacionados}

Esta seção destina-se a análise e discussão da metodologia e dos resultados prospostos por \citeonline{geracao_procedural_jogos_2d,kirillov2019panoptic}. 

\subsection*{Geração Procedural de Mapas para Jogos 2D}

No trabalho \citeonline{geracao_procedural_jogos_2d}, é apresentado uma solução simples para criar mapas de cavernas, calabouços e ilhas para jogos 2D. O algoritmo foi dividido em três partes sendo elas: geração recursiva de terrenos, validação de tamanho e correção da coesão. Os autores concluíram que não existe literatura sobre geração procedural de salas diversas e corredores distintos como o algoritmo proposto. Sugerem duas possibilidades para trabalhos futuros sendo elas: usar algoritmos genéticos para mensurar a qualidade dos mapas gerados e promover pela seleção natural e a outra possibilidade é mesclar o algoritmo proposto com técnicas de geração de salas interligadas por corredores, de forma a possibilitar a criação de mapas com algumas salas pré-definidas inseridas em um mapa aberto contínuo.

\subsection*{Panoptic Segmentation}

No trabalho \citeonline{kirillov2019panoptic} é definido a ideia geral de segmentação panóptica além de definir conceitos importantes como coisas e objetos e a métrica unificada para medir o desempenho de modelos dessa área. Também é feito alguns testes comparando resultados humanos com um modelo simples proposto com eles combinando PSPNet e Mask R-CNN usando a métrica de qualidade panóptica definida por eles. Os resultados mostraram a superioridade humana na segmentação panóptica em três conjuntos de dados diferentes, sendo eles: Cityscapes, ADE20k e Vistas, as métricas usadas foram qualidade panóptica, qualidade semântica, qualidade de reconhecimento, qualidade panóptica
 de coisas e qualidade panóptica de objetos. O melhor resultado para a máquina em comparação com o humano foi no conjunto de dados Cityscapes avaliando a qualidade semântica, sendo 84,1 para o humano e 80,9 para máquina. O pior resultado para a máquina em relação ao humano foi no conjunto de dados ADE20k na qualidade panóptica de coisas, sendo 71,0 para os humanos e 24,5 para a máquina.

 \subsection*{Polygonal Map Generation for Games}

 No artigo de \citeonline{amitp2010} é apresentado toda uma jornada de desenvolvimento de um algoritmo de geração procedural de conteúdo, é mostrando as técnicas para gerar o mapa com o diagrama de Voronoi, gerar os rios e biomas utilizando as camadas de elevação e umidade do polígono e a aplicando isso no diagrama de Whittaker para definir o bioma do polígono. Também é apresentado uma técnica para adicionar ruídos nas arestas dos políginos fazendo com que o mapa se torne mais orgânico e realista.
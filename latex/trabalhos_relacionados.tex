\section{Trabalhos Relacionados}

Esta seção destina-se à análise e discussão da metodologia e dos resultados propostos por \citeonline{kirillov2019panoptic, mohan2020efficientps, amitp2010}.

\subsection*{Segmentação Panóptica}

No trabalho de \citeonline{kirillov2019panoptic}, é definida a ideia geral de segmentação panóptica, além de conceitos importantes como "coisas" e "objetos", e a métrica unificada para medir o desempenho de modelos dessa área. Foram realizados alguns testes comparando resultados humanos com um modelo simples, combinando PSPNet e Mask R-CNN, utilizando a métrica de qualidade panóptica definida por eles. Os resultados mostraram a superioridade humana na segmentação panóptica em três conjuntos de dados diferentes: Cityscapes, ADE20K e Vistas. As métricas usadas foram qualidade panóptica, qualidade semântica, qualidade de reconhecimento, qualidade panóptica de "coisas" e qualidade panóptica de objetos. O melhor resultado para a máquina, em comparação com o humano, foi no conjunto de dados Cityscapes, avaliando a qualidade semântica, sendo 84,1 para o humano e 80,9 para a máquina. O pior resultado para a máquina, em relação ao humano, foi no conjunto de dados ADE20K, na qualidade panóptica de "coisas", sendo 71,0 para os humanos e 24,5 para a máquina.

\subsection*{EfficientPS: Segmentação Panóptica Eficiente}

No trabalho de \citeonline{mohan2020efficientps}, é introduzido o EfficientPS, uma arquitetura concebida para abordar a complexa tarefa de segmentação panóptica, uma área crucial em aplicações como direção autônoma, robótica e realidade aumentada. A metodologia proposta engloba um backbone leve baseado na arquitetura EfficientNet, um módulo de fusão inovador e uma nova função de perda que visa equilibrar as "coisas" e objetos na segmentação. Os resultados obtidos, em comparação com métodos anteriores, utilizando conjuntos de dados reconhecidos como Cityscapes, COCO e KITTI, indicam que o EfficientPS supera implementações anteriores, alcançando pontuações da métrica de qualidade panóptica notáveis. Além disso, destaca-se pela eficiência computacional, evidenciada por tempos de inferência significativamente menores em um único GPU. Este trabalho representa uma contribuição relevante para a pesquisa em segmentação panóptica, fornecendo uma abordagem eficaz e eficiente para aplicações que demandam precisão e agilidade nesta tarefa complexa.

\subsection*{Geração de Mapas Poligonais para Jogos}

No artigo de \citeonline{amitp2010}, é apresentada toda uma jornada de desenvolvimento de um algoritmo de geração procedural de conteúdo. São mostradas as técnicas para gerar o mapa com o diagrama de Voronoi, gerar os rios e biomas utilizando as camadas de elevação e umidade do polígono e aplicando isso no diagrama de Whittaker para definir o bioma do polígono. Também é apresentada uma técnica para adicionar ruídos nas arestas dos polígonos, tornando o mapa mais orgânico e realista.

\section{Trabalhos relacionados}
Esta seção destina-se a análise e discussão da metodologia e dos resultados prospostos por \citeonline{geracao_procedural_jogos_2d}. 

\subsection{Geração Procedural de Mapas para Jogos 2D}

No trabalho \citeonline{geracao_procedural_jogos_2d}, é apresentado uma solução simples para criar mapas de cavernas, calabouços e ilhas para jogos 2D. O algoritmo foi dividido em três partes sendo elas: geração recursiva de terrenos, validação de tamanho e correção da coesão. Os autores concluíram que não existe literatura sobre geração procedural de salas diversas e corredores distintos como o algoritmo proposto. Sugerem duas possibilidades para trabalhos futuros sendo elas: usar algoritmos genéticos para mensurar a qualidade dos mapas gerados e promover pela seleção natural e a outra possibilidade é mesclar o algoritmo proposto com técnicas de geração de salas interligadas por corredores, de forma a possibilitar a criação de mapas com algumas salas pré-definidas inseridas em um mapa aberto contínuo.
